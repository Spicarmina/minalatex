\section{Etimología}

La palabra crédito proviene del latín credititus (sustantivación del verbo credere: creer), que significa "cosa confiada". Así "crédito" en su origen significa entre otras cosas, confiar o tener confianza. Se considerará crédito, el derecho que tiene una persona acreedora a recibir de otra deudora una cantidad en numerario para otros. En general es el cambio de una riqueza presente por una futura, basado en la confianza y solvencia que se concede al deudor. El crédito, según algunos economistas, es una especie de cambio que actúa en el tiempo en vez de actuar en el espacio. Puede ser definido como "el cambio de una riqueza presente por una riqueza futura". Así, si un molinero vende 100 sacos de trigo a un panadero, a 90 días plazo, significa que confía en que llegada la fecha de dicho plazo le será cancelada la deuda. En este caso se dice que la deuda ha sido "a crédito, a plazo". En la vida económica y financiera, se entiende por crédito, por consiguiente, la confianza que se tiene en la capacidad de cumplir, en la posibilidad, voluntad y solvencia de un individuo, por lo que se refiere al cumplimiento de una obligación contraída.